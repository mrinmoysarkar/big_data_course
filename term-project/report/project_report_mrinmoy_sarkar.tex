\documentclass[conference]{IEEEtran}
\IEEEoverridecommandlockouts
% The preceding line is only needed to identify funding in the first footnote. If that is unneeded, please comment it out.
\usepackage{cite}
\usepackage{amsmath,amssymb,amsfonts}
\usepackage{algorithmic}
\usepackage{graphicx}
\usepackage{textcomp}
\usepackage{xcolor}
\graphicspath{{../pdf/}{../jpeg/}{./figures/}}

\def\BibTeX{{\rm B\kern-.05em{\sc i\kern-.025em b}\kern-.08em
    T\kern-.1667em\lower.7ex\hbox{E}\kern-.125emX}}
\begin{document}

\title{Salt Detection Using Segmentation of Seismic Image
%{\footnotesize \textsuperscript{*}Note: Sub-titles are not captured in Xplore and
%should not be used}
%\thanks{Identify applicable funding agency here. If none, delete this.}
}

\author{\IEEEauthorblockN{%1\textsuperscript{st} 
		Mrinmoy Sarkar}
\IEEEauthorblockA{\textit{Department of Electrical \& Computer Engineering} \\
\textit{North Carolina A\&T State University}\\
Greensboro, NC,  USA \\
msarkar@aggies.ncat.edu}
%\and
%\IEEEauthorblockN{2\textsuperscript{nd} Given Name Surname}
%\IEEEauthorblockA{\textit{dept. name of organization (of Aff.)} \\
%\textit{name of organization (of Aff.)}\\
%City, Country \\
%email address}
}

\maketitle

\begin{abstract}
In this project, we present state-of-the-art deep convolution neural network (DCNN) to segment seismic image for salt detection below the earth surface. Detection of salt location is very important for starting mining. Hence, seismic image is used to detect the exact salt location under the earth surface. However, precisely detecting the exact location of salt deposits is very difficult. Therefore, professional seismic imaging still requires expert human interpretation of salt bodies. This leads to very subjective, highly variable renderings. Hence, to create the most accurate seismic images and 3D renderings, we need a robust algorithm that automatically and accurately identifies if a surface target is salt or not. Since, the performance of DCNN is well-known and well-stablished for object recognition in image, DCNN is a very good choice for this particular problem. We successfully applied DCNN to a dataset of seismic images in which each pixel is labeled as salt or not. The result of this algorithm is promissing.
\end{abstract}

\begin{IEEEkeywords}
Seismic Image, Image Segmentation, DCNN, CRF
\end{IEEEkeywords}

\section{Introduction}
A seismic image is produced from imaging the reflection coming from rock boundaries. The seismic image shows the boundaries between different rock types. In theory, the strength of reflection is directly proportional to the difference in the physical properties on either side of the interface. While seismic images show rock boundaries, they don't say much about the rock themselves; some rocks are easy to identify while some are difficult. There are several areas of the world where there are vast quantities of salt in the subsurface. One of the challenges of seismic imaging is to identify the part of subsurface which is salt. However, it is a image segmentation problem from the image processing perspective. There are many robust algorithms available for this task in the literature such as feature-space, image-domain and  physics based techniques \cite{lucchese2001colour}. These techniques have been successfully used for color image segmentation captured from digital camera. Since seismic images are significantly different than digital images, those state-of-the-art techniques fail for segmentation task. There are many challenges for seismic image segmentation. some of them are listed as follows:
\begin{itemize}
	\item Image capturing method.
	\item Uneven distribution of salt and other rocks.
	\item Rock which have density compared to salt.
	\item Only gray-level image means lack of information.
	\item Uneven structure of rocks below the earth surface which causes uneven reflection. 
\end{itemize}

However, there are many state-of-the-art machine learning technique that can be used to solve this problem. The most promissing technique in the literature is deep convolution neural network for any task related to image. This technique has been used for object recognition \cite{ren2015faster}, image segmentation \cite{chen2018deeplab}, style transformation \cite{etemad20093d}, human action recognition \cite{ji20133d}, medical image segmentation \cite{milletari2016v} and image denoising \cite{zhang2017beyond}. Hence, we have used this method to solve the problem at hand. In this work, our contributions are as follows:

\begin{itemize}
	\item We used state-of-the-art DCNN to segment seismic image for salt identification.
	\item We automated the post analysis of seismic images.
	\item We reduced the cost for seismic image analysis.
	\item We relaxed the necessity of human expert for seismic image segmantation.
\end{itemize}

The rest of the paper is organised as in section \ref{Literature survey} literature survey, in section \ref{Methodology} our method to solve the segmentation problem, in section \ref{Experimental Results} experimental results of our method and concludes with conclusion \& future work in section \ref{Conclusion Future Work}.

\section{Literature survey}\label{Literature survey}
\section{Methodology}\label{Methodology}
\section{Experimental Results}\label{Experimental Results}
\section{Conclusion \& Future Work}\label{Conclusion Future Work}











%\begin{table}[htbp]
%\caption{Table Type Styles}
%\begin{center}
%\begin{tabular}{|c|c|c|c|}
%\hline
%\textbf{Table}&\multicolumn{3}{|c|}{\textbf{Table Column Head}} \\
%\cline{2-4} 
%\textbf{Head} & \textbf{\textit{Table column subhead}}& \textbf{\textit{Subhead}}& \textbf{\textit{Subhead}} \\
%\hline
%copy& More table copy$^{\mathrm{a}}$& &  \\
%\hline
%\multicolumn{4}{l}{$^{\mathrm{a}}$Sample of a Table footnote.}
%\end{tabular}
%\label{tab1}
%\end{center}
%\end{table}

%\begin{figure}[htbp]
%\centerline{\includegraphics{fig1.png}}
%\caption{Example of a figure caption.}
%\label{fig}
%\end{figure}



%\section*{Acknowledgment}

%\section*{References}



\bibliographystyle{IEEEtran}
\bibliography{refs}

\end{document}
